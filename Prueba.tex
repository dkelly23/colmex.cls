\documentclass[esp]{colmex}

\institute{Universidad de Procedencia}
\title{Ejemplo}
\subtitle{Clase colmex.cls}
\author{Daniel Kelly}
\mail{djsanchez@colmex.mx}
\date{Septiembre de 2024}

\begin{document}

\maketitle

\chapter{Ejemplo de Capítulo}

\section{Ejemplo de Sección}

\begin{itemize}
    \item Ejemplo de Texto en Color \blue{Azul}.
    \item Ejemplo de Texto en Color \green{Verde}.
    \item Ejemplo de Texto en Color \orange{Naranja}.
\end{itemize}


\section{Comandos de Matemáticas}

\begin{itemize}
    \item Derivada Parcial: $\pd{f(x,y)}{x}$
    \bigskip
    \item Sumatoria: $\dsum{i=1}{N}$
    \bigskip
    \item Producto: $\dprod{i=1}{N}$
    \bigskip
    \item Probability Limit: $\plim{x}$
    \bigskip
    \item Operador de Esperanza: $\E{x|y}$
    \bigskip
    \item Operador de Varianza: $\var{x|y}$
    \bigskip
    \item Underset: $\un{x}{Texto}$
    \bigskip
    \item Negritas: $\mn{x}$
    \bigskip
    \item Números Reales: $\R$
\end{itemize}


\section{Entornos}
\begin{itemize}
    \item \textbf{Ejemplo}

\ej{Título}{
Este es un ejemplo del entorno ejemplo.
}

    \item \textbf{Definición}

\defi{Título}{
Este es un ejemplo del entorno definicion.
}

    \item \textbf{Teorema}

\teo{Titulo}{
Este es un ejemplo del entorno teorema.
}

    \item \textbf{Suposición}

\supo{Título}{
Este es un ejemplo del entorno suposición.
}

    \item \textbf{Demostración}

\proof{
Este es un ejemplo del entorno demostración.
}
\end{itemize}

\section{Figuras}

El comportamiento por defecto de las figuras es ser incluídas al inicio de la página inmediatamente posterior a su inclusión, o en el centro de la siguiente página entera en caso que no haya más texto después.\\

Dentro del entorno \textit{figurenotes} se pueden añadir notas a la figura.

\begin{figure}
    \centering
    \includegraphics[width=0.5\linewidth]{example-image}
    \caption{Título de la Figura}
    \begin{figurenotes}
        asdasdasdasd
    \end{figurenotes}
\end{figure}


\section{Math Arrays}

Simplificamos el uso de los arrays utilizando los comandos \textit{open} y \textit{close}. Automáticamente estos abren un entorno de ecuación, y dentro de él uno de array. En caso que se quiera un array sin que esté dentro de un entorno de ecuación, se recomienda usar el entorno regular. \\

Por defecto, se genera un array con 8 columnas alineadas a la izquierda, en caso que se desee modificar este número, se recomienda hacer uso del entorno regular.

\open
& x_i & = & \x & \text{Ejemplo} \\
\\
& y_i & = & \y & \text{Ejemplo} \\
\close


\end{document}
